%% Towards a complete formalization of PLN
%%
%% Render with: ./render.sh

\documentclass[]{article}
\usepackage{url}
\usepackage{minted}
\usepackage[hyperindex,breaklinks]{hyperref}
\usepackage{breakurl}
\usepackage{listings}
\lstset{basicstyle=\ttfamily\footnotesize,breaklines=false,frame=single}
\usepackage{float}
\restylefloat{table}
\usepackage{longtable}
\usepackage{graphicx}
\usepackage[font=small,labelfont=bf]{caption}
\usepackage[skip=0pt]{subcaption}
\usepackage{circledsteps}
\usepackage{amsfonts}
\usepackage{amsmath}

% For ≞ (requires the LuaLaTeX engine)
\usepackage{unicode-math}
\setmathfont{Stix Two Math}

\begin{document}

\newcommand{\nuPLN}{\nu\textup{PLN}}
%% \newcommand{\Bool}{\mathbb{B}}
\newcommand{\Bool}{\textup{Bool}}
%% \newcommand{\Bool}{B}
\newcommand{\T}{\texttt{True}}
\newcommand{\F}{\texttt{False}}
\newcommand{\Domain}{\mathcal{D}}
\newcommand{\Subsetdom}{\mathcal{S}}
\newcommand{\Predicate}{\mathcal{P}}
\newcommand{\Language}{\mathcal{L}}
\newcommand{\Bernouilli}{\mathcal{B}}
\newcommand{\Bernouillip}{\mathcal{B}_p}
\newcommand{\Field}{\mathcal{F}}
\newcommand{\Model}{M}
\newcommand{\Datax}{D_x}
\newcommand{\DataxD}{D_{x\in\Domain}}
\newcommand{\DataS}{D_{\Subsetdom}}
\newcommand{\STV}[2]{<\!#1, #2\!>}
\newcommand{\limp}{\Rightarrow}
\newcommand{\Nat}{\mathbb{N}}
\newcommand{\anymu}{\hat{\mu}}
\newcommand{\anyomega}{\hat{\omega}}

\title{Towards a Complete Formalization of PLN\\
  \texttt{(DRAFT)}} \author{Nil Geisweiller}
\maketitle

\section{Introduction}
The goal is similar to Solomonoff Universal Induction~\cite{TODO},
that is we want to approach a first order (unknown but computable)
distribution $\mu$ given observations, using a second order
(uncomputable but known) distribution $\nu$, called the Universal
Distribution.  In Solomonoff Induction, observations are bit strings
produced by a Turing machine\footnote{Note that even though the sample
space of $\nu$ is made of deterministic Turing machines, $\nu$ can
approximate any computable distribution $\mu$ (thus non-deterministic)
by maintaining an ensemble of such Turing machines.}.  In PLN however,
observations are outcomes from an indexed boolean random variable,
representing the outputs of evaluating a predicate on some inputs.
Such predicate is called the \emph{observable predicate}.  In practice
PLN allows multiple observable predicates however one can assume one
predicate without loss of generality.  Indeed, to emulate multiple
predicates, one can introduce an extra component in the predicate's
domain to ``select'' the predicate of interest.  Also, since it is
observed by a random variable, such predicate is not necessarily
deterministic (though it could be).  As such, one may think of the
observable predicate as being a program drawn from a certain
Probabilistic Programming Language.  In the following section we
formally define the above.

Because this reformulation of PLN departs somewhat from the definition
of PLN in the PLN book~\cite{TODO}, we give it a new name, $\nuPLN$.

\section{Definitions}
In its original form, PLN purposely avoids relying on an underlying
global probability distribution.  I am not against this in principle.
I will simply admit that I cannot conceive a complete definition of
PLN that does not rely on such global probability distribution.  I
would also point out that a publication released after the PLN book by
the principal authors of the PLN book, Ben Goertzel and Matt Ikle,
very much aligns with the idea of a global probability
distribution~\cite{TODO}, and was in fact a great source of
inspiration for writing this very document.  The next subsection is
dedicated to define the global probability distribution which $\nuPLN$
is intended to derive from.

\subsection{Global Probability Distribution}
\label{sec:globalprob}
Let $(\Omega, \Field, \nu)$ be a probability space such that
\begin{itemize}
\item $\Field$ is the event space, a $\sigma$-algebra on $\Omega$.
\item $\nu : \Field \to [0, 1]$ is a universal distribution, further
  defined below.
\item $\Omega$, the set of possible worlds, is the sample space
  associated to $\nu$, such that each element $\anyomega \in \Omega$
  contains
  \begin{enumerate}
  \item a probabilistic predicate $\anymu \in \Language$ over a domain
    $\Domain$, described in a probabilistic programming language
    $\Language$,
  \item a mapping of $\anymu$ from $\Domain$ to Boolean.  For instance
    if $\Domain$ is $\Nat$, then a possible mapping could be
    $(\anymu\ 0) = \T,\ (\anymu\ 1) = \F, \ (\anymu\ 2) = \T, \dots$,
    corresponding so far to a predicate indicating the evenness of a
    natural number.  Note that the world $\anyomega$ contains
    the entire, potentially infinite, mapping, even though in reality
    an observer can only have access to a finite subset of it.
  \end{enumerate}
\end{itemize}
An observer lives in a particular world $\omega \in \Omega$, called
the \emph{true world}, which includes the \emph{true generator} or
\emph{true predicate}, $\mu$, and the \emph{true history}, which is
the complete mapping of evaluations of $\mu$ over the domain
$\Domain$.  Note that since $\mu$ is probabilistic, it does not
deterministically determine the history, thus the true history is just
one possible history among an infinity of histories compatible with
$\mu$.  Note that throughout the document $\omega$ and $\mu$ refer to
the true world and true generator respectively, which is to be
contrasted with $\anyomega$ and $\anymu$ which refer to arbitrary
elements of $\Omega$ and $\Language$ respectively.  Although when it
is clear that the definitions are generic and apply to any arbitrary
world and the true world alike, we will use $\omega$ and $\mu$ as
well.  Since $\mu$ is a probabilistic predicate, its type signature
cannot merely be
$$\mu : \Domain \to \Bool$$ where $\Bool = \{\F, \T\}$.  To capture
its probabilistic nature we give it the following type signature
$$\mu : \Domain \to \Omega \to \Bool$$ in a curried fashion.  Meaning
that given its argument, it produces a boolean random variable.
Therefore the observable predicate can be viewed as an indexed boolean
random variable.  Of course, $\mu$ never gets to be evaluated on a
different world than the one it belongs to, thus we can write
$(\mu\ a)$ while meaning $(\mu\ a\ \omega)$, but we still need to keep
the $\Omega$ argument in order to reason about possible worlds since
the true world is unknown.  Also, in cases where the domain $\Domain$
can be decomposed into multiple components, a curried notation will be
used interchangeably.  So instead of
$$\mu : (\Domain_1 \times \dots \times \Domain_n) \to \Omega \to \Bool$$
the following
$$\mu : \Domain_1 \to \dots \to \Domain_n \to \Omega \to \Bool$$ will
be used.  This will be convenient to express inheritance relationships
between partially applied predicates.  As already hinted, the
application of $\mu$ to an input $x$ is denoted with the traditional
functional programming style
$$(\mu\ x)$$ Thus if the domain is decomposed into subdomains
$\Domain_1$ to $\Domain_n$, applying $\mu$ to all its inputs $x_1$ to
$x_n$ will be denoted
$$(\mu\ x_1\ \dots\ x_n)$$ Likewise for the $\omega$ argument
$$(\mu\ x_1\ \dots\ x_n\ \omega)$$ Such functional programming
notation is used for $\mu$ because it is currying-friendly.  For the
rest, we keep using the traditional mathematical function application
style, such as
$$\nu(E)$$ denoting the application of the probability distribution
$\nu$ to the event $E$.  In case $\mu$ is known to be deterministic,
$\Omega$ could potentially be dropped, but that is not going to be our
working assumption for now.  Additionally to the functional
programming style, we may use parametric notation, so for instance
instead of
$$(\mu\ x_1\ x_2)$$ we may write
$$(\mu_{x_1}\ x_2)$$ where $x_1$ is viewed as a parameter and $x_2$ is
viewed as the argument of $\mu_{x_1}$.\\
With that, let us now define key random variables to access $\Omega$:
\begin{itemize}
\item $\Model : \Omega \to \Language$ with measurable space
  $(\Language, \Field_{\Language})$, where $\Language$ is a certain
  probabilistic programming language and $\Field_{\Language}$ is a
  $\sigma$-algebra on $\Language$.  Thus, $\Model$ takes a world
  $\anyomega \in \Omega$ and outputs the probabilistic program $\anymu
  \in \Language$ generating that world.  Note that this random
  variable is inaccessible from an observer within that world.  An
  observer within that world only has access to a finite record of
  evaluations of $\anymu$.  However, this random variable is important
  to reason about multiple worlds, we suspect it is particularly
  important for higher order reasoning.
\item $\DataxD : \Omega \to \Bool$, a Boolean random variable indexed
  by values in $\Domain$.  Unlike $\Model$, $\DataxD$ is at least
  partially accessible from an observer within that world.  Meaning,
  such observer can gather data for a finite subset $\Subsetdom$ of
  $\Domain$.  In this case $\DataS$ represents a finite family of
  Boolean random variables, corresponding the set of accessible
  observations.  $\Model$ and $\DataxD$ are related by the following
  equality
  $$(\anymu\ x\ \anyomega) = (\Datax\ \anyomega)$$ where $\anyomega
  \in \Omega$ such that $(M\ \anyomega) = \anymu$.  Or simply, in
  curried fashion
  $$(\anymu\ x) = \Datax$$
\end{itemize}
Due to the equality above, the distribution of observations is
entirely determined by a model $\anymu$.  In other words, it suffices
to define a distribution over $\Language$, the prior distribution over
possible models, to define $\nu$ (as far as $M$ and $\Datax$ are
concerned anyway).  Then, relating observations to models can be done
using regular Bayesian inference.  The prior is defined by
$$\nu(M \in L)$$ where $L \in \Field_{\Language}$.  Note how it is
expressed in terms of elements of $\Field_{\Language}$, instead of
elements of $\Language$.  It is because, for the purpose of recovering
PLN with the Bayesian approach, $\Language$ needs to be continuous.  I
will explain this in detail, but for now let us use this notation to
formulate Bayes' theorem
$$\nu(M \in L | \DataS) = \frac{\nu(M \in L) \times \nu(\DataS | M \in
  L)}{\nu(\DataS)}$$ where $\Subsetdom$ is a finite subset of
$\Domain$.  For the sake of simplicity, we may drop $M$ since it is
the only random variable that ranges over $\Language$, and simply
write
$$\nu(L | \DataS) = \frac{\nu(L) \times \nu(\DataS |
  L)}{\nu(\DataS)}$$
An example of prior will be given in~\ref{TODO},
but in general it can be viewed as a parameter of $\nuPLN$.  That is,
given a certain prior of $\nu$ over $\Language$, one can derive a
certain flavor of $\nuPLN$.  Since $(\mu\ x) = \Datax$, a history data
point will be represented as $(\mu\ x) = \T$ or $(\mu\ x) = \F$, and
the usage of the random variable $\Datax$, and especially $\DataS$,
will be reserved in the formulation of $\nu$ to refer to the observed
history.

\subsection{PLN Global Distribution vs Solomonoff Induction}
Note that unlike with Solomonoff Universal Induction, $\anymu$
represents a probabilistic predicate rather than a computable
probability function calculating the probability of any event,
although the latter can be derived from the former.

The \emph{true distribution} in Solomonoff Induction corresponds to
the \emph{real predicate} here.  We prefer to use the word \emph{real}
when denoting objective reality, rather than \emph{true} because
\emph{true predicate} could be understood as a predicate that always
outputs true.

\section{PLN and the Global Distribution}
In Section~\ref{sec:derules} will proceed to derive PLN rules from the
global probability distribution introduced in
Section~\ref{sec:globalprob}, but for now let us recall important
notions of PLN and how to relate them to the global distribution
defined in Section~\ref{sec:globalprob}.

\subsection{Concepts vs Predicates}
The PLN book describes respectively the notions of \emph{concepts} and
\emph{predicates}, and their respective associated relationships
\emph{inheritance} and \emph{implication}.  As explain in Section 2.6
\emph{Higher-Order Logical Relationships} of the PLN book, there is a
perfect isomorphism between concepts and inheritance on one side and
predicates and implication on the other side.  Concerned with
conciseness, we will pick a side, the predicate side, and essentially
forget about the other side, the concept side, for the rest of this
document.  But before we do so let us recall what is a concept, the
inheritance between two concepts, and the isomorphism between concepts
and predicates.

\subsubsection{Concepts and Inheritance}
A concept is a fuzzy (or, as I prefer to say, probabilistic, for
reasons I will explain in Section~\ref{TODO}) set, and the
\emph{extensional} inheritance between two concepts is a
probabilitized subset relationship.  Originally, in the PLN book, the
inheritance relationship is defined as an explicit mixture of
extensional and intensional inheritances. We will show however that
they are in fact both the same thing, the extensional inheritance is a
way to approach inheritance solely via \emph{induction}, and
intensional inheritance is a way to approach inheritance solely via
\emph{abduction}.  Any one side, extensional or intensional, is good
enough to define the other, so let us pick one, the extensional side,
and define inheritance with it.  The extensional inheritance between
two concepts can be viewed as a probabilitized subset relationship.
It allows to express things like ``most members of a set are also
members of another set''.  For instance one could express in PLN that
90\% of birds fly, with
$$(\texttt{Inheritance}\ \texttt{Bird}\ \texttt{Fly}) \measeq 0.9$$
Such knowledge might have been obtained by observing a finite sample
of birds and whether or not they fly.  Meaning there could be an
uncertainty on the 90\% itself.  To represent such uncertainty PLN
uses a second order distribution, in this case a Beta distribution as
it is an ideal choice to represent the posterior of the parameter of a
Bernouilli distribution given observations.  Under this assumption
only two numbers are required to determine the parameters, $\alpha$
and $\beta$, of the associated Beta distribution.  A \emph{truth
value} called \emph{simple truth value} was created for this purpose
and is thus described by two numbers: a strength (a proxy for a
probability estimate) and a confidence over this strength from which
the $\alpha$ and $\beta$ parameters of the Beta distribution can be
recovered.  For instance, given a simple truth value one may express
that 90\% of birds fly with a confidence of 0.99
$$(\texttt{Inheritance}\ \texttt{Bird}\ \texttt{Fly})\ \measeq\ \STV{0.9}{0.99}$$
where 0.9 is the strength and 0.99 is the confidence.  The confidence
is a value between 0 and 1 that actually encodes the sample size that
was used to obtained the strength.  The higher the sample size, the
higher the confidence.  More information about that will be provided
in Section~\ref{TODO} but for now let us just leave it at that as it
is enough for what we are concerned with in this Section which is the
isomorphism between concepts and predicates.

\subsubsection{Isomorphism between Concepts and Predicates}
To every concept one can associate a predicate and vice versa.  To go
from concept to predicate one can use the \emph{indicator function},
and to go from predicate to concept one can use the \emph{satisfying
set}.  These notions are well known for crisp predicates and sets and
thus will not be detailed any further here.  The only difference in
PLN is that concepts are probabilistic, meaning that a probability (or
potentially a second order probability) can be attached to the
membership of an element to a concept.  Likewise, predicates are
probabilistic in the sense that a (second order) probability can be
attached to the evaluation of an argument to a predicate.  As one
would expect the isomorphism also applies between the inheritance
relationship on the concept side and the implication relationship on
the predicate side.  So for instance, on the predicate side one can
express the same inheritance relationship between birds and fly as
follows
$$(\texttt{Implication}\ \texttt{IsBird}\ \texttt{DoesFly})\ \measeq\ \STV{0.9}{0.99}$$
where $\texttt{IsBird}$ and $\texttt{DoesFly}$ have been obtained by
taking the indicator functions of $\texttt{Bird}$ and $\texttt{Fly}$.
As mentioned earlier, we will drop the concept side and only focus on
the predicate side for the remaining of the document.

\subsection{Relating Predicates to \texorpdfstring{$\mu$}{𝜇}}
As explained in Section~\ref{sec:globalprob} there is only one global
predicate, $\mu$.  To manipulate multiple predicates we can simply
introduce a additional parameter in the domain of $\mu$ indicating
predicates.  Let us call this subdomain $\Predicate$ which ranges over
symbols representing predicates, in this case $\mu$ may have the type
signature
$$\mu : \Predicate \to \Domain \to \Omega \to \Bool$$ So for instance
to express that a cat is a mammal, traditionally represented as
$$\textit{Cat} \limp \textit{Mammal}$$ one may use the following
parametric notation of $\mu$
$$\mu_{\textit{Cat}} \limp \mu_{\textit{Mammal}}$$ Likewise, to
represent the evaluation of the predicate \textit{Cat} over a certain
cat instance, $\textit{cat}_{123}$, one may use
$$(\mu_{\textit{Cat}}\ \textit{cat}_{123})$$ corresponding to
the traditional representation
$$(\textit{Cat}\ \textit{cat}_{123})$$ where $\textit{cat}_{123} \in
\Domain$.

\subsection{Truth Value as Posterior Distribution}
\label{sec:truthvalue}

\subsection{Statement versus Judgement}
\label{sec:statjudge}
Like in NAL, a PLN \emph{statement} designates a logical statement
without truth value, such as
$$P \limp Q$$ While a PLN \emph{judgment} designates a PLN statement
with a truth value attached to it, such as
$$P \limp Q\ \measeq\ \STV{0.9}{0.8}$$

\section{Deriving PLN Rules}
\label{sec:derules}
Our goal here is to derive every PLN rules in the PLN book from the
global distribution that has been defined in
Section~\ref{sec:globalprob}.  By doing so we hope not only to
provide a clear unambiguous definition for each rule, but also an
ideal to approach as using a global distribution should give us the
means to derive a convergence theorem a la Solomonoff.

\subsection{Predicate Direct Introduction}
This rule is meant to calculate the truth value corresponding to a
Predicate from direct observations.  Its truth value can be understood
as the marginal probability of a predicate to be true, irrespective of
the inputs.  It is a subcase of the Implication Direct Introduction
presented in Section~\ref{sec:impldirect} where the implicant is the
\emph{Universal Predicate} (a predicate that is always true), but is
described here as its own rule to simplify the presentation.  Indeed,
it should be easier to understand the Implication Direct Introduction
rule after understanding the Predicate Direct Introduction rule.

To derive the predicate direct introduction rule we consider the
posterior probability of a Bernouilli process with parameter $p$ given
all available observations of the predicate in question.  Let us begin
with $\mu$ itself.  In order to formulate this posterior we need to
assume that $\Language$ has a Bernouilli sampler in its set of
operators.  Let $\Bernouilli$ be the subset of programs of $\Language$
consisting of a single Bernouilli call, thus comprised of the
following programs
\begin{minted}{scheme}
  (Bernouilli p)
\end{minted}
where \mintinline{scheme}{p} is the parameter of the Bernouilli
distribution ranging over $[0, 1]$.  Such probabilistic program, when
executed, outputs $\T$ with a probability of \mintinline{scheme}{p} or
$\F$ with a probability of 1 - \mintinline{scheme}{p}.  Let
$\Field_{\Bernouilli}$ be a Borel $\sigma$-algebra on $\Bernouilli$.
Let us assume that $\Field_{\Bernouilli}$ is a subfield of
$\Field_{\Language}$.  We claim that that if $\mu$ is restricted to
the programs in $\Bernouilli$ then the posterior of
\mintinline{scheme}{p} corresponds to the truth value of $\mu$,
expressed as follows
$$\nu(L_{\Bernouilli} | \DataS) = \frac{\nu(L_{\Bernouilli}) \times
  \nu(\DataS | L_{\Bernouilli})}{\nu(\DataS)}$$ where $L_{\Bernouilli}
\in \Field_{\Bernouilli}$.  NEXT

\subsection{Implication Direct Introduction}
\label{sec:impldirect}
This rule is not explicitly stated as such in the PLN book but can be
derived from iteratively applying induction and revision, and reflects
the formula of extensional inheritance/implication given in Section
2.4.1 \emph{The Semantics of Inheritance} of the PLN book, at least
the strength part.  To obtain the confidence part we assume that the
second order distribution is a Beta distribution, like in the Section
4.2 \emph{From Imprecise Probabilities to Indefinite Probabilities} of
the PLN book.  In order to derive a Beta distribution as second order
distribution we can assume an underlying Bernouilli process with
parameter $p$ with a Beta distribution as prior.  The Jeffreys prior
where $\alpha=\beta=\frac{1}{2}$ is often the default choice.  Given
the PLN statement
$$P \limp Q$$ Let us express its truth value as the posterior
probability of the parameter $p$ of the underlying Bernouilli process.
In order to map a Bernouilli process onto an implication we zoom-in to
the data points where $P$ is true.

NEXT

\section{conclusion}

\bibliographystyle{splncs04}
\bibliography{local}

\end{document}
