%% Towards a complete formalization of PLN
%%
%% Render with: ./render.sh

\documentclass[]{article}
\usepackage{url}
\usepackage{minted}
\usepackage[hyperindex,breaklinks]{hyperref}
\usepackage{breakurl}
\usepackage{listings}
\lstset{basicstyle=\ttfamily\footnotesize,breaklines=false,frame=single}
\usepackage{float}
\restylefloat{table}
\usepackage{longtable}
\usepackage{graphicx}
\usepackage[font=small,labelfont=bf]{caption}
\usepackage[skip=0pt]{subcaption}
\usepackage{circledsteps}
\usepackage{amsfonts}

% For ≞ (requires the LuaLaTeX engine)
\usepackage{unicode-math}
\setmathfont{Stix Two Math}

\begin{document}

\newcommand{\nuPLN}{\nu\textup{PLN}}
%% \newcommand{\Bool}{\mathbb{B}}
\newcommand{\Bool}{\textup{Bool}}
%% \newcommand{\Bool}{B}
\newcommand{\T}{\top}
\newcommand{\F}{\bot}
\newcommand{\Domain}{\mathcal{D}}
\newcommand{\Subdomain}{\mathcal{S}}
\newcommand{\Language}{\mathcal{L}}
\newcommand{\Event}{\mathcal{F}}
\newcommand{\Model}{M}
\newcommand{\Data}{D_{x\in\Domain}}
\newcommand{\DataS}{D_{\Subdomain}}
\newcommand{\STV}[2]{<\!#1, #2\!>}

\title{Towards a Complete Formalization of PLN\\
  \texttt{(DRAFT)}} \author{Nil Geisweiller}
\maketitle

\section{Introduction}
The goal is similar to Solomonoff Universal Induction~\cite{TODO},
that is we want to approach a first order (unknown but computable)
distribution $\mu$ given observations, using a second order
(uncomputable but known) distribution $\nu$, called the Universal
Distribution.  In Solomonoff Induction, observations are bit strings
produced by a Turing machine\footnote{Note that even though the sample
space of $\nu$ is made of deterministic Turing machines, $\nu$ can
approximate any computable distribution $\mu$ (thus non-deterministic)
by maintaining an ensemble of such Turing machines.}.  In PLN however,
observations are outcomes from an indexed boolean random variable,
representing the outputs of evaluating a predicate on some inputs.
Such predicate is called the \emph{observable predicate}.  In practice
PLN allows multiple observable predicates however one can assume one
predicate without loss of generality.  Indeed, to emulate multiple
predicates, one can introduce an extra component in the predicate's
domain to ``select'' the predicate of interest.  Also, since it is
observed by a random variable, such predicate is not necessarily
deterministic (though it could be).  As such, one may think of the
observable predicate as being a program drawn from a certain
Probabilistic Programming Language.  In the following section we
formally define the above.

Because this reformulation of PLN departs somewhat from the definition
of PLN in the PLN book~\cite{TODO}, we give it a new name, $\nuPLN$.

\section{Definitions}
In its original form, PLN purposely avoids relying on an underlying
global probability distribution.  I am not against this in principle.
I will simply admit that I cannot conceive a complete definition of
PLN that does not rely on such global probability distribution.  I
would also point out that a publication released after the PLN book by
the principal authors of the PLN book, Ben Goertzel and Matt Ikle,
very much aligns with the idea of a global probability
distribution~\cite{TODO}, and was in fact a great source of
inspiration for writing this very document.  The next subsection is
dedicated to define the global probability distribution which $\nuPLN$
is intended to derive from.

\subsection{Global Probability Distribution}
\label{subsec:globalprob}
Given an observable predicate $\mu$ with domain $\Domain$, let
$(\Omega, \Event, \nu)$ be a probability space such that
\begin{itemize}
\item $\Event$ is the event space, a $\sigma$-algebra on $\Omega$.
\item $\nu : \Event \to [0, 1]$ is a universal distribution, further
  defined below.
\item $\Omega$ is the sample space associated to $\nu$, such that each
  element $\omega \in \Omega$ contains a model $\mu$\footnote{Those
  familiar with Solomonoff Universal Induction, please note that here
  $\mu$ represents a probabilistic predicate rather than a computable
  probability function calculating the probability of any event,
  although the latter can be derived from the former.}, a
  probabilistic predicate over a certain domain $\Domain$, as well as
  the complete mapping of $\mu$ from $\Domain$ to its Boolean
  co-domain.
\end{itemize}
Since $\mu$ is a probabilistic predicate, its type signature cannot
merely be
$$\mu : \Domain \to \Bool$$ where $\Bool = \{\F, \T\}$.  To capture
its probabilistic nature we give it the following type signature
$$\mu : \Domain \to \Omega \to \Bool$$ in a curried fashion.  Meaning
that given its argument, it produces a boolean random variable.
Therefore the observable predicate can be viewed as an indexed boolean
random variable.  Of course, $\mu$ never gets to be evaluated on a
different world than the one it belongs to, but we still need to keep
the $\Omega$ argument in order to reason about possible worlds since
the true underlying world is unknown.  Also, in cases where the domain
$\Domain$ can be decomposed into multiple components, a curried
notation will be preferred.  So instead of
$$\mu : (\Domain_1 \times \dots \times \Domain_n) \to \Omega \to \Bool$$
the following
$$\mu : \Domain_1 \to \dots \to \Domain_n \to \Omega \to \Bool$$ will
be used.  This will be convenient to express inheritance relationships
between partially applied predicates.  Applying $\mu$ to an input $x$
will be denoted with the following functional programming notation
$$(\mu\ x)$$ Likewise, $\mu$ applied to more than two arguments will
be denoted
$$(\mu\ x\ \omega)$$ and so on.  For $\mu$ such functional programming
notation is used because it is currying-friendly.  For the rest, we
keep using the traditional mathematical function application notation,
such as
$$\nu(E)$$ denoting the application of the probability distribution
$\nu$ to the event $E$.  In case $\mu$ is known to be deterministic,
$\Omega$ could potentially be dropped, but that is not going to be our
working assumption for now.  With that, let us now define key random
variables to access $\Omega$:
\begin{itemize}
\item $\Model : \Omega \to \Language$ with measurable space
  $(\Language, \Event_{\Language})$, where $\Language$ is a certain
  probabilistic programming language and $\Event_{\Language}$ is a
  $\sigma$-algebra on $\Language$.  Thus, $\Model$ takes a particular
  world $\omega \in \Omega$ and outputs the probabilistic program $\mu
  \in \Language$ generating that world.  Note that this random
  variable is inaccessible from an observer within that world.  An
  observer within that world only has access to a finite record of
  evaluations of $\mu$.  However, that random variable will be
  convenient to define aspects of the semantics of $\nuPLN$.
\item $\Data : \Omega \to \Bool$, a Boolean random variable indexed by
  values in $\Domain$.  Unlike $\Model$, $\Data$ is at least partially
  accessible from an observer within that world.  Meaning, such
  observer can gather data for a finite subset $\Subdomain$ of
  $\Domain$.  In that case $\DataS$ represents a finite family of
  Boolean random variables, corresponding the set of accessible
  observations.  $\Model$ and $\Data$ are related by the following
  equality
  $$(\mu\ x\ \omega) = (D_x\ \omega)$$ where $\omega$ runs over all
  elements of $\Omega$ such that $(M\ \omega) = \mu$.  Or simply, in
  curried fashion
  $$(\mu\ x) = D_x$$
\end{itemize}
Due to the equality above, the distribution of observations is
entirely determined by a model $\mu$.  In other words, it suffices to
define a distribution over $\Language$, the prior distribution over
possible models, to define $\nu$ (as far as $M$ and $\Data$ are
concerned anyway).  Then, relating observations to models can be done
using regular Bayesian inference.  The prior is defined by
$$\nu(M \in L)$$ where $L \in \Event_{\Language}$.  Note how it is
expressed in terms of elements of $\Event_{\Language}$, instead of
elements of $\Language$.  It is because, for the purpose of recovering
PLN with the Bayesian approach put forward, $\Language$ needs to be
continuous.  I will explain that aspect in detail, but for now let us
use this notation to formulate Bayes' theorem
$$\nu(M \in L | \DataS) = \frac{\nu(M \in L) \times \nu(\DataS | M \in
  L)}{\nu(\DataS)}$$ where $\Subdomain$ is a finite subset of
$\Domain$.  An example of prior will be given in~\ref{TODO}, but in
general it can be viewed as a parameter of $\nuPLN$.  That is, given a
certain prior of $\nu$ over $\Language$, one can derive a certain
flavor of $\nuPLN$.

\section{Concepts vs Predicates}
The PLN book describes respectively the notions of \emph{concepts} and
\emph{predicates}, and their respective associated relationships
\emph{inheritance} and \emph{implication}.  There is a perfect
isomorphism between concepts and inheritance on one side and
predicates and implication on the other side.  Concerned with
conciseness, we will pick a side, the predicate side, and essentially
forget about the other side, the concept side, for the rest of this
document.  But before we do so let us recall what is a concept, the
inheritance between two concepts, and the isomorphism between concepts
and predicates.

\subsection{Concepts and Inheritance}
A concept is a fuzzy (or, as I prefer to say, probabilistic, for
reasons I will explain in Section~\ref{TODO}) set, and the
\emph{extensional} inheritance between two concepts is a
probabilitized subset relationship.  Originally, in the PLN book, the
inheritance relationship is defined as an explicit mixture of
extensional and intensional inheritances. We will show however that
they are in fact both the same thing, the extensional inheritance is a
way to approach inheritance solely via \emph{induction}, and
intensional inheritance is a way to approach inheritance solely via
\emph{abduction}.  Any one side, extensional or intensional, is good
enough to define the other, so let us pick one, the extensional side,
and define inheritance with it.  The extensional inheritance between
two concepts can be viewed as a probabilitized subset relationship.
It allows to express things like ``most members of a set are also
members of another set''.  For instance one could express in PLN that
90\% of birds fly, with
$$(\texttt{Inheritance}\ \texttt{Bird}\ \texttt{Fly}) \measeq 0.9$$
Such knowledge might have been obtained by observing a finite sample
of birds and whether or not they fly.  Meaning there could be an
uncertainty on the 90\% itself.  To represent such uncertainty PLN
uses a second order distribution, in this case a Beta distribution as
it is an ideal choice to represent the posterior of the parameter of a
Bernouilli distribution given observations.  Under this assumption
only two numbers are required to determine the parameters, $\alpha$
and $\beta$, of the associated Beta distribution.  A \emph{truth
value} called \emph{simple truth value} was created for this purpose
and is thus described by two numbers: a strength (a proxy for a
probability estimate) and a confidence over this strength from which
the $\alpha$ and $\beta$ parameters of the Beta distribution can be
recovered.  For instance, given a simple truth value one may express
that 90\% of birds fly with a confidence of 0.99
$$(\texttt{Inheritance}\ \texttt{Bird}\ \texttt{Fly})\ \measeq\ \STV{0.9}{0.99}$$
where 0.9 is the strength and 0.99 is the confidence.  The confidence
is a value between 0 and 1 that actually encodes the sample size that
was used to obtained the strength.  The higher the sample size, the
higher the confidence.  More information about that will be provided
in Section~\ref{TODO} but for now let us just leave it at that as it
is enough for what we are concerned with in this Section which is the
isomorphism between concepts and predicates.

\subsection{Isomorphism between Concepts and Predicates}
To every concept one can associate a predicate and vice versa.  To go
from concept to predicate one can use the \emph{indicator function},
and to go from predicate to concept one can use the \emph{satisfying
set}.  These notions are well known for crisp predicates and sets and
thus will not be detailed any further here.  The only difference in
PLN is that concepts are probabilistic, meaning that a probability (or
potentially a second order probability) can be attached to the
membership of an element to a concept.  Likewise, predicates are
probabilistic in the sense that a (second order) probability can be
attached to the evaluation of an argument to a predicate.  As one
would expect the isomorphism also applies between the inheritance
relationship on the concept side and the implication relationship on
the predicate side.  So for instance, on the predicate side one can
express the same inheritance relationship between birds and fly as
follows
$$(\texttt{Implication}\ \texttt{IsBird}\ \texttt{DoesFly})\ \measeq\ \STV{0.9}{0.99}$$
where $\texttt{IsBird}$ and $\texttt{DoesFly}$ have been obtained by
taking the indicator functions of $\texttt{Bird}$ and $\texttt{Fly}$.
As mentioned earlier, we will drop the concept side and only focus on
the predicate side for the remaining of the document.  If the reader
still wishes to understand more clearly this isomorphism the PLN book
covers it well.

\section{Deriving PLN Rules}
Our goal here is to derive every PLN rules in the PLN book from the
global distribution that has been defined in
Section~\ref{subsec:globalprob}.  By doing so we hope not only to
provide a clear unambiguous definition for each rule, but also an
ideal to approach as using a global distribution should give us the
means to derive a convergence theorem a la Solomonoff.

\subsection{Implication Direct Introduction}
This rule is not explicitly stated as such in the PLN book but it can
be derived from iteratively applying induction and revision, and
directly reflects the formula of extensional inheritance/implication

\section{conclusion}

\bibliographystyle{splncs04}
\bibliography{local}

\end{document}
